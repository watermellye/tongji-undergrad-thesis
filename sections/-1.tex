\section{总结与展望}\label{sec:-1}

\subsection{总结}

本课题对人类食道医学造影视频进行分析,标注出不同浓度钡餐在食道内的分布区域。本课题所涉及的数据样本规模较小,目标形态差异较大,且对标注精度要求高。经查阅,目前尚缺相关的研究成果与算法模型,研究内容具有创新性,难度较大。本课题运用多种医学图像处理方法,以及计算机视觉和图形学领域的方法,结合自己的优化,完成了该工程实践与挑战。

本课题处理的数据为头颈部侧位X线检查所得的DR图像经短时连续采集而得到的视频。本课题输出的数据为与输入数据同规格的视频,其中每一帧为钡餐范围及浓度标定图像。在研究中,我们对每组输入数据依次进行以下处理:
\begin{enumerate}
    \item 使用自动窗宽窗位算法,标准化输入数据。
    \item 使用光流法对视频数据进行配准,尽可能抵消受试者的头部整体运动,以使后续对钡餐运动矢量的提取更准确。
    \item 对视频中的每一对相邻帧图像进行处理,获取一组代表本底值的像素点。具体步骤如下:
    \subitem 对相邻帧图像求差,得到差值灰度图。
    \subitem 使用双边滤波对灰度图降噪。
    \subitem 对降噪后图像行二值化。
    \subitem 使用膨胀腐蚀等形态学操作对二值图进一步降噪。
    \subitem 使用铃木算法从二值图中获取轮廓。
    \subitem 使用格林公式计算各轮廓所包围的面积,快速过滤噪点小轮廓,保留感兴趣的大轮廓。
    \subitem 对每个过滤后留下的轮廓,使用泛洪法填充算法获取轮廓所包围的像素点的集合。
    \subitem 基于相邻两帧的信息,赋予集合中的每个像素点本底值。
    \item 为进一步利用视频所包含的运动信息,对于每个新求得的本底值蒙版,使用之前得到的蒙版进行扩张操作。
    \item 将扩张后的蒙版与当前帧数据进行运算,得到每一帧的钡餐区域和相对浓度标定。
    \item 将每一帧标定结果组合成连续帧,得到输入数据同规格的视频并输出。
\end{enumerate}

由于缺乏标准集,我们自定义了一套以人工观察为主要依据的评价指标。分析结果得到以下结论:本研究方法对喉部的标定结果尚可,对口部的标定。错误主要集中在对钡餐区域的标记范围明显溢出,以及口部数据中对非钡餐区域出现了误标。我们总结问题并提出改进方案,但由于个人精力和能力等原因,在本科阶段的研究暂止步于此。

\subsection{展望}

当前的标定结果的准确度尚不足以进入实际使用。在接下来的研究中,我们首先尝试实现\cref{sec:-2} 中对实验结果分析后所提出的各个改进方案以提高钡餐标定准确度。我们需获取更多的数据集,以及经医学专家手动标注的标准集。

若项目进展顺利,后续我们还将面临更艰难的挑战。我们需通过不同粘稠度的钡餐在食道内流动形成的封闭几何区域来确定会厌软骨的位置,并追踪其在吞咽过程中连续的运动轨迹。接下来,我们尝试标定钡餐被会厌软骨罩住的容积,以及侧漏发生时气管所处的部位与会厌软骨相对位置。随后,我们将基于流动参数和运动形态变化,结合动力学模型,实现三维模型的重建。最后,我们需对不同粘稠度的钡餐的运动过程进行三维仿真,从而为医师提供可视化的诊断治疗辅助手段。
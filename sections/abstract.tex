\MakeAbstract{
    食道疾病在我国民众中的患病率逐年攀升,针对食道疾病的预防和诊断显得尤为关键。食道造影作为一种简便的检测方法,能够直观地展示食道状况,有效探测食道结构的异常。当前针对DR图像的研究较多,但对DR视频的研究较少。本课题对人类食道医学造影视频进行分析,面临数据样本规模较小,目标形态差异较大,标注精度要求高等难点。我们提出了一种自动调整窗宽窗位的方法,使关注的钡餐区域更为突出,并采用光流法对视频进行逐帧配准。我们采用降噪、轮廓提取、腐蚀膨胀等图像处理方法,获取了视频数据中每一帧图像里的钡餐区域及其本底值。最后,我们设计了一种拓展逐帧处理成果的方法,依据多帧序列生成最终的钡餐区域及相对浓度标定结果。
}{X线钡餐造影,视频处理,图像分割}

\MakeAbstractEng{
    The prevalence of esophageal diseases in our population is increasing year by year. Therefore, prevention and diagnosis of esophageal diseases are particularly critical. As a simple detection method, esophagogram can visualize the condition of the esophagus and effectively detect abnormalities in esophageal structures. In this project, we deeply analyze the video and images of human esophagogram, facing the difficulties of small data sample size, large variation of target morphology, and high requirement of annotation accuracy. We propose a method to automatically adjust the window width and window level to make the barium meal region more prominent, and use the optical flow method to align the video frame by frame. We used image processing methods such as noise reduction, contour extraction, and erosion expansion to obtain the barium meal region and its background value in each frame of the video data. Finally, we designed a method to extend the frame-by-frame processing results to generate the final barium meal region and relative concentration calibration results based on multi-frame sequences.
}{X-ray barium meal, video processing, image segmentation}

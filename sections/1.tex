\section{绪论}\label{sec:1}

\subsection{研究背景及意义}

随着社会的进步和生活水平的不断提高,人们的生活节奏加快,饮食习惯和生活方式改变,小酌喝酒、暴饮暴食等都可能对食道产生不良影响。加之疾病因素、遗传因素等原因,导致食道疾病在国人中的发病率逐年上升。食道疾病包括食道炎、食道裂孔疝、食道癌等,在现代社会已成为常见病患问题之一,对患者的生活质量和身体健康带来较大影响。因此,针对食道疾病的预防和诊断显得尤为重要。

食道造影作为一种检查手短,能对食道进行直观显示,有效地检测食道结构异常。食道医学造影图像分析的研究对于食道疾病的诊断、评估治疗方案和监测疗效具有重要意义。在进行食道造影操作时,对硫酸钡浓度的关注至关重要,以免过低或过高影响疾病的检出率。针对不同患者的吞咽能力,应该配置适宜黏稠度和浓度的造影剂。然而在实际检查过程中,医生常常依赖主观判断造影剂的黏稠度,缺失客观的参考标准,致使患者在接受检查时可能面临误吸和呛咳等风险。

本课题旨在通过对人类食道医学造影视频与图像的深入分析,准确标识不同浓度钡餐在食道内的分布区域,并运用计算机图形学相关算法,构建流动钡餐在双重视角的封闭几何范围。结合后续课题,这项研究将有助于提高食道疾病的诊断准确性、降低误诊或漏诊率,并能对疾病的评估和治疗方案提供数据支持。此外,借助计算机技术,能够进一步推动食道影像诊断技术的发展,为解析更多健康隐患和降低医疗成本奠定基础,帮助患者对自身食道疾病的认识和预防增加信心。

\subsection{国内外研究现状}

医学影像技术在食道检测方面发挥着重要作用,其中 X 射线便是医疗界常见的初步筛查方法之一。X 射线始于 1859 年,由德国物理学家、诺贝尔物理学奖得主威廉·康拉德·伦琴发现\cite{johansen1996circumstances}。自 X 射线被发现后,其在医学影像诊断技术中的潜在应用便迅速被确认,并广泛应用于全身各组织器官的检查。在 X 线造影检查过程中,人体不同器官与组织的密度与厚度差异导致其在影像上呈现出自然的黑白层次对比效果。如今,利用 CR(计算摄影)系统(核心部件为 IP(成像板))的技术,光信号可转换成电信号,进而以电子数据形式储存。X 射线影像中展现的信息(包括不同明暗/黑白对比的像素)质量与数量取决于被照射对象的因素(例如原子序数、密度、厚度)以及射线本身的诸多因素(例如线质、线量、散射线等)。

在许多情景下,为了使造影图像更加清晰,需要使用对人体无害的造影剂。对于钡、碘等元素,由于其原子系数较大,与 X 射线发生光电效应时会产生特征 X 射线。这些射线能够穿透人体组织,形成灰雾效果的影像,使显示对比度提高,达到理想的检测效果。这种检查方法在临床医学上被称为 X 线钡餐造影检查。钡餐造影,亦即上消化道钡剂造影检查,是让受检者摄取糊状硫酸钡(显影剂)后,在 X 线照射下查看消化道是否存在病变的检测方法。硫酸钡既不溶于水,也不溶于脂质,因此不会被胃肠道黏膜吸收,对人体基本上无毒性。以果仁、小鱼刺为例,一些透 X 线的异物无法直接被显影,这些异物被称为食管阴性异物,需通过钡剂造影方能诊断。较大的异物在吞服钡剂后,在透视下可见钡剂流向异物处受阻,形成分流或偏流绕过异物下行。此外,细小异物也可通过向钡剂中加入棉絮来检测,在透视下观察到钡絮钩挂,即使在吞咽或饮水操作后,钡絮仍不下行,可指示异物所在位置。

\subsubsection{静态图像分析}

本课题首先应对医学视频数据的逐帧画面进行静态分析。尽管目前国内外在数字化X线钡餐造影领域的研究颇为丰富,然而在钡餐范围标定以及浓度方面的探讨却相对不足。据查阅,仅有2021年一项涉及此领域的研究成果。在该成果中,吴越等学者探讨了不同钡餐浓度及增稠剂浓度对患者吞咽舒适度的影响\cite{wu2021}。

有鉴于此,我进一步检索了关于其他医学影像分割或标注方面的研究。国内外关于医学影像的分割与标注领域研究繁盛,包括传统方法及深度学习方法两大发展方向。传统的分割技巧有基于模糊聚类法、基于区域生长法、基于形态学分水岭法、基于可变形模型法、基于异常检测法、基于图谱匹配法等。在众多方法中,分水岭算法与区域生长分割以其精准、迅速、高效的特点而备受青睐。分水岭算法最初由 Digabel 和 Lantuejoul 率先引入图像处理领域\cite{wang2009watershed},而由柴\cite{chai2007watershed}等提出的改进版分水岭算法,尽管在抗噪声方面有了显著提升,但仍不免出现过分割的现象等。基于深度学习的分割方法主要使用全卷积神经网络(FCN)\cite{2015Fully},结合U-Net\cite{Ronneberger2015}或Dense-Net\cite{huang2017densely}网络进行优化。

由于钡餐浓度随时间变化大,且在不同器官中的形态变化大,传统的分割方法均无法取得较好的效果。而基于深度学习的分割方法主要面临缺乏公开数据集的问题。同时,食道X线钡餐造影图像涉及多方位和多器官,难以将网络进行迁移学习、训练和应用。

\subsubsection{动态图像分析}

想要更加准确地描绘钡餐分布区域以及其浓度变化,还需结合连续时间片所带来的流动参数信息及运动形态变化信息。目前,国内外关于二维医学视频的连续标注研究较少。2020年,刘强等人研究了冠脉造影图序列的生理运动补偿减影,通过分解生理运动信号,在不依靠心电门控的前提下完成帧间配准\cite{liu2020}。在其它领域,当前针对视频目标跟踪的研究主要聚焦于小样本学习或弱数据标注,无监督学习或缺陷数据集,多目标或特定目标跟踪,以及提升速度和稳定性等方面。

本课题所涉及的数据样本规模较小,目标形态差异较大,且对标注精度要求高。根据以上分析,目前相关领域尚缺乏成熟研究成果与稳定的算法模型。本课题的研究内容具有创新性,前置研究较少,面临一定挑战。

\subsection{论文的结构安排}

本文的研究重点是对输入的医学影像数据进行逐帧标注,准确描绘钡餐分布区域以及其浓度变化。为更出色地完成本课题研究,我们采用多种医学及图形学算法,包括:窗口调节、图像配准、图像去噪,以及轮廓的提取、计算及形变等技术。最后,我们提出了一种基于图像处理的算法,充分利用时域信息,使用前驱帧处理结果扩展当前帧处理结果。

本课题所采用的各组数据均存储于独立的DIC格式文件中,数据类型为DICOM医学影像。DICOM(数字成像和通信标准)是一种在放射学、心血管以及其他医学领域广泛应用的医学影像格式,其文件中包含影像以及患者、设备和诊断信息等元数据。在本课题研究中,每组数据对应一次独立的X线检查,以连续时间片的形式记录了钡餐在受试者体内的二维图像。图像为受试者的侧位头部及颈部。在每次检查过程中,研究人员会通过针筒将定量钡餐注入受试者口腔。钡餐可能出现于受试者的口喉、咽部或食道;钡餐可能在口腔静止,被吞咽或被吐出。

本文结构如下:

第一章:绪论

本章首先阐述了本文的研究背景和意义。本课题对X线检查生成的医学视频数据进行标注,描绘钡餐分布区域及浓度。本章综合概括了与本课题相关的算法在国内外的研究现况。最后,以论文的组织结构为线索,对本文的主要内容进行阐述。

第二章:医学视频预处理与优化

本章将对待处理的视频数据进行预处理。首先运用自动窗宽窗位算法,凸显出我们所关注的钡餐区域;接着从数据中提取运动形态变化信息,并采用光流法对视频进行逐帧配准,为后续逐帧图像分析处理奠定基础。

第三章:逐帧图像处理研究

本章应用多种图像处理算法,提取视频数据中每一帧图像里的钡餐区域及其本底值。操作流程如下:对相邻帧图像求差值,得到待处理图像;对图像进行去噪处理;使用算法获取图像轮廓及每个轮廓所包围的像素数;循环运用腐蚀和膨胀算法筛选并剔除小轮廓以实现进一步去噪;对上述步骤得到的钡餐区域进行本底值标注,生成最终蒙版。

第四章:基于多帧序列生成标定结果

本章中,我们提出一种拓展逐帧操作成果的算法。针对每一帧图像,以上述获得的蒙版作为基础,借助前驱帧所得蒙版进行扩张操作;在此过程中,不断维护各蒙版的有效性,以提高处理速度。最终,我们得到每一帧中的钡餐区域及相对浓度标定。

第五章:实验结果分析

本章首先将原始数据、蒙版、最终成果同时展示,观察并分析区域标定的准确性。由于原始采集仪器数据及环境数据暂缺,无法确定钡餐的绝对浓度,因此采用相对浓度,并以R通道值形式叠加于原始数据进行展示,以便观察和分析浓度标定的准确性。
\section{简介}\label{sec:introduction}

图像分割是图像处理的一个重要研究领域,特别是在医疗应用方面。随着X射线计算机断层扫描(CT)和磁共振成像(MRI)的发展,三维图像的采集成为可能,并允许对人体内部进行科学研究。然而,要使这些三维图像成为诊断或定量分析的有效帮助,必须事先对其进行处理,以提取与解剖学相关的部分。这种分割是一个非常重要的步骤,它决定了所有后续过程的准确性。在这项技术中,图像特征与同质性标准配合使用,以形成分割区域。

许多作者都对区域增长技术感兴趣:Zucker (1976)\cite{zucker1976region}是第一个发表现有方法的详尽清单的人,几年后Haralick和Shapiro (1985)\cite{haralick1985survey}提出根据像素的联系组合对它们进行分类。最简单的区域生长方法包括合并两个相邻的像素 $P_x$ 和 $P_y$,如果它们的灰度 $f(P_x)$ 和 $f(P_y)$ 差距不是太远。更确切地说,$|f(P_x)-f(P_y)| \leq \alpha$,其中 $\alpha$ 是一个固定阈值。这种方法由于其简单性而具有吸引力,但是 $\alpha$ 的选择需要了解被处理图像的灰度等级以获得良好的结果。此外,它可能会导致连锁效应,特别是对于具有低对比度形状边界的图像或具有光照变化的图像。这种方法可以通过对相邻像素的合并增加一个条件来改进(Sivewright and Elliot, 1994\cite{sivewright1994interactive}; Sekigushi and Sano, 1994\cite{sekigushi1994interactive}). 如果一个像素 $P_x$ 和一个相邻的像素 $P_y$ 之间的灰度等级 $f(P_x)$ 和 $f(P_y)$ 相差不大,并且 $f(P_x)$ 属于某个灰度等级范围,那么该像素将合并到同质区域的一个相邻的像素上。这个想法可以用以下公式表示:$|f(P_x)-f(P_y)| \leq \alpha$ 和 $|f(P_x)-f(P_0)| \leq \beta$,其中 $P_0$ 是区域生长的起点。然而,这种方法高度依赖于 $\alpha$ 和 $\beta$ 的选择,当它们的值过大时,就会产生超出需要分割的区域的泄漏。此外,产生的分割区域是连接的。因此,如果不在每个不相连的组件中放入初始种子,一个由许多不相连的组件组成的物体将不会被分割。

Revol和Jourlin (1997)在\cite{revol1997new}以前的工作中提出过一种新的方法,在这个方法中,基于一个能使差异最小化的特殊的扩张过程,每一步都重新考虑生长区域的像素分配。由初始种子创建的区域可能是不相连的,也可能不包含这个种子。然而,所产生的分割总是取决于调整的同质性阈值。

在许多系统中,初始参数的调整导致了结果的不可重复性,这阻碍了它们在一些医疗应用中的使用,因为在这些应用中必须从分割中计算出定量参数。在本文中,我们提出了一种新的自动区域生长算法(ARG)来分割三维图像。没有任何初始参数,如最大的同质性阈值或种子的位置需要调整。其原理是通过将最大同质性阈值从一个很小的值增加到一个很大的值来建立一个区域生长序列,并使用评估函数(assessment function)来确定序列中的最佳区域生长。

我们的分割方法将在下一节中详细介绍。在\cref{sec:assessment_function}中,我们提出了基于边界或区域方法的评估函数,并在人工图像上进行了测试。在\cref{sec:results_and_evaluation}中,我们将方法应用于三维核磁共振图像,进行三维小梁骨分割并分析了结果,并与自动和手动阈值进行了比较。

\documentclass{tongjithesis}
\usepackage{tongjithesis}

%%%%%%%%%%%%%%%%%%%%%%%%%%%%%%%%%%%
% 若你想要用句点(.)替换句号(。)
% 可以打开下面的注释
%%%%%%%%%%%%%%%%%%%%%%%%%%%%%%%%%%%
% \catcode`\。=\active
% \newcommand{。}{\ifmmode\text{.}\else .\fi}
%%%%%%%%%%%%%%%%%%%%%%%%%%%%%%%%%%%

%%%%%%%%%%%%%%%%%%%%%%%%%%%%%%%%%%%
% 在Windows上使用时:
% 若你想避免将 Python 加入环境变量,可以在此手动指定python环境
% 打开下面的注释,并替换尖括号<>中的路径
% e.g. {<X:/your/path/to/python>/python.exe} -> {C:/software/Anaconda3/envs/latex/python.exe}
%%%%%%%%%%%%%%%%%%%%%%%%%%%%%%%%%%%
% \newcommand{\myPython}{<X:/your/path/to/python>/python.exe}
% \renewcommand{\MintedPygmentize}{\myPython\space -m pygments}
%%%%%%%%%%%%%%%%%%%%%%%%%%%%%%%%%%%

\newcommand{\myPython}{C:/software/Anaconda3/envs/latex/python.exe}
\renewcommand{\MintedPygmentize}{\myPython\space -m pygments}

\begin{document}

\school{电子与信息工程学院}
\major{计算机科学与技术}
\student{1950084}{陈泓仰}
\thesistitle{基于评估函数的自动三维区域生长算法}{}
\thesistitleeng{Automated 3D region growing algorithm based on an assessment function}{}
\thesisadvisor{}
\thesisdate{2023}{5}{19}

\MakeCover


\pagestyle{firststyle}
\MakeAbstract{
    本文提出了一种用于三维图像的自动分割的新的区域生长算法。无需设置同质性阈值或种子位置等初始参数。此算法的原理是通过将最大的同质性阈值从很小的数值增加到很大的数值来建立一个区域生长序列。在每个分割的区域上,我们用一组经测试图像验证过的三维参数来评估分割的质量。这组参数被称为评估函数,用来确定最佳的同质性标准。本算法在三维核磁共振图像上进行了测试,用于分割骨小梁样本,以便对骨质疏松症进行量化。与自动和手动阈值的比较表明,我们的算法表现得更好。它的主要优点是消除了由于噪声造成的孤立点,并保留了骨结构的连接性。
}{图像分割,区域生长,同质性标准,三维MR成像}

\MakeAbstractEng{
    A new region growing algorithm is proposed for the automated segmentation of three-dimensional images. No initial parameters such as the homogeneity threshold or the seeds location have to be adjusted. The principle of our method is to build a region growing sequence by increasing the maximal homogeneity threshold from a very small value to a large one. On each segmented region, a 3D parameter that has been validated on a test image assesses the segmentation quality. This set of values called assessment function is used to determine of the optimal homogeneity criterion. Our algorithm was tested on 3D MR images for the segmentation of trabecular bone samples in order to quantify osteoporosis. A comparison to automated and manual thresholding showed that our algorithm performs better. Its main advantages are to eliminate isolated points due to the noise and to preserve connectivity of the bone structure.
}{Image segmentation, Region growing, Homogeneity criterion, 3D MR imaging}


\clearpage
\tableofcontents   %放置目录
\clearpage

\pagestyle{mainstyle}
\input{sections/01_intro}
\clearpage
\input{sections/02_math}
\clearpage
\input{sections/03_reference}
\clearpage
\input{sections/04_float}
\clearpage
\input{sections/05_conclusion}
\clearpage


\addcontentsline{toc}{section}{参考文献}
\bibliography{note}

\clearpage
\section*{谢\ 辞}
\addcontentsline{toc}{section}{谢辞}

这项工作属于法国国家科学研究中心(CNRS)的PRC-GdR ISIS和GDR 2237研究小组的科学课题范围。作者非常感谢法国里昂CPE核磁共振实验室的Pr. André Briguet先生提供的三维磁共振图像。他们还感谢Olivier Beuf博士对MR图像的采集。

\end{document}
